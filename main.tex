\documentclass{article}
\usepackage[utf8]{inputenc}
\usepackage{graphicx}
\usepackage{cmap}
\usepackage{dirtytalk}
\usepackage{listofsymbols}
\usepackage[table]{xcolor}
\usepackage{array}
\usepackage{enumitem}
\usepackage{fancyhdr}%
\usepackage{lipsum}% 
\usepackage[margin=1in]{geometry}

\fancyhf{}% Clear all headers/footers
\fancyhead[C]{UNCLASSIFIED // FOUO}
\fancyfoot[C]{\thepage}
%\cfoot{\thepage}
\pagestyle{fancy}
\thispagestyle{plain}

\title{\includegraphics[width=2in]{Pics/86esb.png}\\
\\
\vspace{1.5cm}
\textbf{\LARGE TACTICAL STANDARD OPERATING PROCEDURES (TACSOP)}}

\author{\Large \textbf{86th EXPEDITIONARY SIGNAL BATTALION}}
\date{ }

\usepackage{natbib}
\usepackage{graphicx}

\begin{document}

\maketitle

\newpage
\section*{CARD 101 – MISSION STATEMENT}
Mission: Provide Command and Control (C2) and oversee the Engineering, Installation, Operation, Maintenance, and Defense (EIOM-D) of state-of-the-art communications to enable US and Multinational forces to dominate Unified Land Operations.
\\
Vision: Train, develop, and maintain a culture of warriors and leaders who seize the initiative, are agile in thought and action, and ready to provide expeditionary communications in the most challenging environments.
\\
Intent: The 86th number one priority remains actionable READINESS through the execution of training, the maintenance of expeditionary capabilities, caring for Soldiers and Families, leader development, and the preparation for Signal Support Missions.

\newpage
\section*{CARD 102 – COMPANY ORGANIZATION}
?
\begin{figure}[h!]
\centering
\includegraphics[scale=1.7]{universe}
\caption{The Universe}
\label{fig:universe}
\end{figure}

\newpage
\section*{CARD 103 – RESPONSIBILITIES}
\begin{enumerate}
\item Company Roles.
\begin{enumerate}
\item COMMANDER (CDR).
\begin{enumerate}
\item Responsible for everything the Company does or fails to do.
\item Operate on Company CMD and BN CMD.
\item Positions himself where he can best command and control the Company. 
\item Integrate and synchronize available combat multipliers.
\item Accomplish all missions assigned to the Company in accordance with the BN Commander’s intent.
\item Preserve the fighting capability of the Company.
\end{enumerate}
\item NETWORK OPERATIONS OFFICER (NETOPS OFFICER).
\begin{enumerate}
\item Maintain communications link between the Teams, Company, and Battalion on CO, BN CMD and BN O and I nets.
\item Forward all reports from Company to Battalion.
\item Post and maintain current situation map.  Analyze situation to determine.
\begin{enumerate}
\item Enemy disposition, strength, and composition
\item Likely enemy COA
\item Friendly locations/actions
\item Status of network and switches
\end{enumerate}
\item Act as second in command.  “Paint the picture” of the network to the CDR and to BN. Be prepared to immediately assume command of the company and continue the mission.  
\item Maintain current status of all organic and attached personnel and equipment.  Keep the CDR informed and request assistance through the BN A and L or CMD nets.  
\item Serve as the 1SG’s link to the battlefield.
\item Serve as the CDR’s second set of eyes in reviewing plans and analyzing situations.
\item Assist the 1SG in planning and coordinating the Company CSS.
\end{enumerate}
\item FIRST SERGEANT (1SG)
\begin{enumerate}
\item NCOIC of the Company. Trains Soldiers for all planned and unplanned missions.
\item Coordinate all resupply.  Plan, coordinate and execute all LOGPAC functions.
\item Supervise evacuation of casualties, KIAs, and EPWs.
\item Supervise movement of replacements from trains to platoons.
\item Submit personnel reports and logistical reports to include location of all broken vehicles on BN A and L.
\item Operate on BRT CMD and BDE A and L.
\item Responsible for consolidation of platoons LOGSTAT/PERSTAT JBC-P data and forwarding all stats to BN ALOC.
\end{enumerate}
\item Platoon Leader - The platoon leader directs the operations of the JNN platoon.  From pre-deployment reconnaissance to operational management, their primary responsibility is to install and operate the JNN and the extension nodes (i.e. CPN, Data Package, and S2CIP) linked to it (regardless of parent unit) as well as direct the defense of the JNN site. The platoon leader is responsible for all electronic and logistical support for his/her JNN, as well as for every CPN, Data Package, and S2CIP connected to it. The CPN supervisor will check on the CPNs, enabling the platoon leader to know the status of the node at all times.  The platoon leader ensures that required reports are sent to Company NETOPs in accordance with this SOP.  The NC is under the direct operational control of the BN NETOPs during operation.  Ensures that site standards are high.
\item Platoon Sergeant - As the NCOIC of the JNN site, his primary responsibility is for the operation of his/her JNN and the care of his soldiers.  During combat operations and periods of restricted movement, physical inspections of assigned CPNs will be severely restricted. Nevertheless, he checks on the morale and welfare of soldier’s daily by VOIP if necessary. Shares all of the responsibilities assigned to the JNN platoon leader. Either the platoon sergeant or the platoon leader will be in the JNN Ops area at all times.
\item Switch Supervisor - The NCOIC of the JNN switch. He/she directs the installation and operation of the JNN and directs modifications to Promina and router configurations (with approval of the Net Tech) as required accomplishing the mission. The Switch Supervisor is the technical director of the site.  From the JNN Operations area the supervisor and crew provide the tech control for the installation and troubleshooting of subordinate CPN links.
\item LOS Supervisor - The LOS supervisor is responsible for the operations of the LOS system, as well as the quality of all radio paths. The LOS Supervisor recommends the optimum positioning and use of the LOS radio assets at the JNN site.
\item CPN Supervisor – This NCO is responsible for the installation and operation of the CPN and directs modifications to router configurations (with approval of the Net Tech) as required accomplishing the mission.  The CPN Supervisor is the technical director of the site.  
\end{enumerate}
\end{enumerate}

\newpage
\section*{CARD 104 – BASIC LOADS}
\begin{enumerate}
\item General. The following is a table that outlines the recommended basic load amounts for all small arms within a company for each soldier. Ex. Soldier with M4 is authorized 210 rounds. EX. Soldier with M9 is authorized 30 Rounds. Companies total round authorization will change based on MTOE and assignment under a division or Task Force.
INSERT TABLE
\item Scope. This card explains CL V operations.
\begin{enumerate}
\item CLASS V OPERATIONS: Resupply will occur during evening LOGPAC but emergency resupply may be requested using the YELLOW 5 report. CLV prepacks will be used when appropriate.
\begin{enumerate}
\item CL V prepacks are:
\begin{enumerate}
\item Pack Type of Unit Remarks
\item A131 7.62 4:1 28,800 1 pallet
\item A059 5.56 Ball 1680 1 case
\item A363 9mm 2000 1 case
\item DODIC Nomenclature Quantity Remarks
\item A131 7.62 4:1 28,800 1 pallet
\item A059 5.56 Ball 5040 3 cases
\item A064 5.56 Link 4:1 4800 3 cases
\item G881 Grenade, Fragment 90 3 cases
\item B546 40mm HE 72 1 case
\item B508 40mm SMK 36 1/2 case
\item G900 Grenade, Thermite 30 1 case (in trailer)
\end{enumerate}
\end{enumerate}
\end{enumerate}
\end{enumerate}

\newpage
\section*{CARD 105 – TROOP LEADING PROCEDURES}
\begin{enumerate}
    \item The Troop Leading Procedures (TLP) provide team level leaders a guideline for planning and executing training. The eight steps of the TLP can be remembered using the mnemonic device RIMICCIS (ri-mi-kiss). Knowing the TLP steps, and knowing where you are as a leader in your planning process will ensure mission success. 

\item Leaders must remain engaged in all steps of this model. PLT leaders and PLT sergeants should ensure team leaders and section sergeants are present for a “terrain walk” RECON if their signal assemblage is part of an operation. Receiving a team leader’s input on where to establish their system will ultimately lead to a greater level of personal involvement at the team level. When leaders at all levels are incorporated into planning, the plan takes on more value as the inputs are more diverse. 
\item Similarly, PLT leaders and PLT sergeants should not leave elements of step 8 (Supervise and Refine) solely as the responsibility of the lowest level leader. Although it is ultimately the team chief’s responsibility to ensure their Soldiers and equipment are ready for training, the PLT’s senior leadership and Co level leadership will also reap benefits of remaining thoroughly engaged in supervising and refining training.

\item Troop leading procedures.
\begin{enumerate}
    \item Receive the Mission: task identification – specified/implied/essential, limitations and constraints, additional resources required, coordinate requirements, reverse planning schedule

\item Issue the WARNO: enemy situation, restated mission, changes to task organization, delegation of critical tasks, begin PCC/PCI, coordination requirements, reverse planning schedule, time and place the OPORD will be issued, service support requirements, 
\item Make a Tentative Plan: Identify task and purpose based on mission, METT-TC analysis, integration of IPB, analysis and COA, contingencies
\item Initiate Movement: Time and route to the SP, determine why/when to move, position the company at a location advantageous to mission preparation
\item Conduct RECON: Map/air/ground, terrain walk, leader’s RECON
\item Complete the Plan: Complete the details of how each subordinate unit accomplishes tasks, integrate fire support (FS) or engineering (EN) plan as required, develop communications plan.
\item Issue the Order: Use terrain model/sand table if time/resources permit, state at a minimum: mission, CDR’s intent, task and purpose, concept of the operations, ROE, MEDEVAC/CASEVAC procedures
\item Supervise and Refine: Conduct PCC/PCI, supervise section/team leader checks, conduct Co/PLT level back brief, conduct Co/PLT level rehearsals, check and conduct training on critical tasks, ensure subordinates have completed implied tasks. 
\item Back brief Format: Back briefs are normally conducted after the OPORD is published, and after rehearsals. Back brief should include: BDE/BN CDRs intent, mission statement, PLT task and purpose by phase/event, key actions, risk assessment, safety/tactical/fratricide risks, and risk controls/mitigation/reduction.
	\end{enumerate}
	\end{enumerate}
	
\newpage
\section*{CARD 106 – INDIVIDUAL EQUIPMENT PACKING LIST}
\begin{enumerate}
    \item General. This card prescribes field training packing list for Worn, Rucksack, Assault Pack and A Bag Spring /Summer and B Bag Fall /Winter. Additional items are added as prescribed by the commander.

\item Duty uniform will be OCPs, unless otherwise specified.  Personal comfort items may be included in the list, subject to weight allowances.  Credit cards, i.e., government/personal credit cards, driver’s license, pagers, ID tags, civilian/military identification card and eyeglasses are some items that should be in your possession while deploying.  “A” and “B” bag packing lists are shown below. (Will vary depending on season and location)
\\
INSERT EXCEL SHEET
\end{enumerate}

\newpage
\section*{CARD 107 – RUCKSACK SETUP}
\begin{enumerate}
    \item General. This card prescribes setup of Rucksack and Assault Pack. Models not shown will adhere to same marking standards. Reflective belt will be centered horizontally on Rucksack as prescribed by commander.
\\
INSERT IMAGES
\end{enumerate}

\newpage
\section*{CARD 108 – VEHICLE AND ISU 90 MARKING PROCEDURES}
\begin{enumerate}
    \item General.   This card outlines the procedures for vehicle markings on use in civilian roads and highways.

\item Administrative Markings.
\begin{enumerate}
    \item Apply administrative markings to identify equipment ownership and alert personnel to safety and operating restrictions in flat black paint. 
\begin{enumerate}
    \item Use 2” stencils for front, rear, and nickname markings.
\item No vehicles are allowed to have caricatures, logos, unit emblems, animals, skulls/crossbones, sabers, or qualifications ratings stenciled or placed on the vehicles.  
NOTE.   During combat operations, remove name and rank markings or cover them in green tape.
\end{enumerate}
\end{enumerate}

\item Numbering System.
\begin{enumerate}
    \item The left side or passenger side when viewed from the front will have “86 ESB” and the right side or driver’s side will have the new bumper number scheme number, as an example, “A100”  (keep in mind, this is the unit identifier:  HQ, HC, A, B, C, followed by the new number).   When looking from the back, the left side or driver’s side will have “86 ESB” and the right side or passenger side will have the new bumper number, as an example “A100”. 
\end{enumerate}
\item Numbering system (Examples)
\begin{enumerate}
    \item 	The figure below shows sample battle boards for the 86th ESB:
\\
INSERT IMAGES
\\
\item The figure below shows the ISU 90 Container for the 86th ESB:
\\
INSERT IMAGES
\end{enumerate}

\item Convoy Flags: Each serial and marching unit must be identified and marked.  The following applies:
\begin{enumerate}
    \item Lead Vehicle:
    \begin{enumerate}
        \item Mark each column, serial, and unit with flags that are 12x18 \item The lead vehicle will carry a blue flag
\end{enumerate}
\item Convoy Commander Vehicle:
\begin{enumerate}
    \item Mark each column, serial, and unit with flags that are 12x18 inches. 
\item The convoy commanders for the column will carry a white and black flag.  All flags will be mounted on the front left of the vehicle and must not interfere with the driver’s vision or any functional component of the vehicle. 
\end{enumerate}
\item Trail Vehicle:
\begin{enumerate}
    \item Mark each column, serial, and unit with flags that are 12x18 inches. 
\item The trail vehicle will carry a green flag. 
\end{enumerate}
\item	Convoy Signs
\begin{enumerate}
    \item The first vehicle (pacesetter) in each element of the convoy must have on its front a sign with 4-inch black letters on a yellow background reading CONVOY FOLLOWS. The last vehicle of each convoy element will have on the rear a sign reading CONVOY AHEAD. CONVOY AHEAD signs are not on maintenance or medical vehicles unless the vehicle's purpose is to represent the end of the convoy. 
\item The material used for these signs will be yellow reflective paint or sheeting, 8” X 50”, with a 3/8” black border:  The letters should be 4” in height, non-reflective black, and centered.
\end{enumerate}
\item	Safety and Warning Devices
\begin{enumerate}
    \item 	A rotating amber warning light will be placed on cranes (wreckers), oversize or overweight vehicles, and the first and last vehicles in a convoy. The lights will be on at all times when the convoy is operating outside a military installation.
\item Convoy vehicles will also display reflective L-shaped symbols 12 inches long and 2 inches wide at the lower corners of the vehicle's body.
\item While moving at night or during periods of reduced visibility, lead, trail, and oversize/overweight vehicles will operate four-way flashers. 
\item Headlights of all vehicles moving in convoy or halted on road shoulders must be on low beam at all times. While halted on shoulders, vehicles equipped with emergency flasher systems must also have these lights operating. 
\end{enumerate}
\item	Safety Equipment
\begin{enumerate}
    \item An approved fire extinguisher. 
\item An approved first aid kit. 
\item One set (pair) of tire chains when snow or ice conditions may be encountered. 
\item An approved highway warning kit.
\end{enumerate}
\item	CONVOY CONTROL NUMBER:
\begin{enumerate}
    \item Each vehicle in a convoy must have a convoy control/clearance number.  This number will identify the convoy during the entire movement. It will be placed on both sides and, if possible, on the front and rear of each vehicle. It will also be placed on the top or hood of the head and trail vehicles to insure identification from the air. Numbers are normally written in chalk and must be written large enough to be read from a short \item Typical Control Number: The CCN has 10 digits. The first two digits identify the location (post or state) from which the convoy originates. The next four digits represent the Julian date. The next three digits are the sequence number, followed by a single digit, designating the type of movement. The type of movement designators are as follows:
    \begin{enumerate}
    \item S - Outsize/Overweight Vehicles
\item E – Explosives
\item	H - Hazardous Cargoes
\item	C - All Other Convoys
    \end{enumerate}
\end{enumerate}
\end{enumerate}
\item	Limited Visibility Guidelines
\begin{enumerate}
    \item Vehicles: While moving at night or during periods of reduced visibility, lead, trail, and oversize/overweight vehicles will operate four-way flashers. 
    \item Convoy vehicles will also display reflective L-shaped symbols 12 inches long and 2 inches wide at the lower corners of the vehicle's body (refer to AR 55-29). 
\end{enumerate}
\end{enumerate}

\newpage
\section*{CARD 109 – PRE-COMBAT CHECKS (PCC) / PRE-COMBAT INSPECTIONS (PCI}
\begin{enumerate}
    \item Purpose.  To establish general guidelines and procedures for pre-deployment operations.

\item Responsibilities.  Commanders are responsible for ensuring their personnel are familiar with the information contained in this annex.

\item Procedures.
\begin{enumerate}
    \item Leaders will conduct PCCs in preparation for pre-deployment inspections
	\item Leaders will conduct PCIs to ensure all Soldiers and equipment are ready for deployment.
	\item The packing list is the guideline for what all Soldiers should have for deployment. Leaders can add to the packing list as necessary.
\end{enumerate}
\item Pre-Combat Checks (PCC)
\begin{enumerate}
    \item On Vehicle:
    \begin{enumerate}
         
\item	Blank 2404s					
\item	Blank Sector Sketch Cards (for site defense)
\item	Technical Manuals (TMs) for all weapon systems and equipment
\item Maps (includes strip map) and Graphics
\item	Thorough PMCS complete (according to the TM)
\item Dispatch
\item Batteries for Lights, DAGR, NVGs, COMSEC, etc.
\item Tool Bag (complete)
\item Pioneer Tools (complete)
\item Fire Extinguishers
\item First-aid kit
\item Garbage bags
\item Toilet paper
\item All fuel cans full
\item Water cans full
\item	Commo Plan Card, Radios, Frequencies, extra hand-mics (H-350), etc.
\item	Tow hooks and cables
\item	MREs (72Hrs)
\item Battlefield Recognition Markers (BRM)
\item Basic Issue Items (BII)
\end{enumerate}
\item Soldier Equipment:
\begin{enumerate}
     
\item ACH Army Combat Helmet with cover, NVG mount and Soldiers name and battle roster displayed on helmet band
\item Load carrying/barring equipment (IOTV/FLC) to include canteens, ammo pouches, and first-aid pouch
\item MEDEVAC Card
\item	ID Tags and Card
\item	Driver’s License (military)
\item	Issued weapon and AMMO
\item Eye protection
\item	Gloves
\item	Hearing protection
\item	Lost Sensitive Items Procedures card
\item	Protective mask complete with inserts if needed 
\item	Individual equipment including: issued weapons, NVGs, and DAGRs
\item	Mission-oriented Protective Posture (MOPP) gear or equivalent. (If not worn, should be stored in the C-Bag according to packing list)
\item	An CBRNE warning and reporting Graphic Training Aids (GTA 3-6-8)
\item	2 x full canteens of water/camelbacks
\item	Valid Military Driver’s License
\item	Notebook, pens, and pencils
\end{enumerate}
\item Leaders Equipment:
\begin{enumerate}
    \item	A complete map with current overlays
    \item	Alcohol or waterproof markers (black, blue, green, red)
    \item	Current TACSOP
    \item	Light and weather data
    \item	A sensitive items serial-number checklist (radios, masks, weapons, NVGs and DAGRs)
    \item	9 line MEDDAVAC Request (GTA Card)
\end{enumerate}
\item Vehicle Equipment:
\begin{enumerate}
\item	Ammunition storage
\item	Crew bags (ruck sacks, A bags, and CBRNE, if needed and B bag if needed)
\item	Water cans
\item	Fuel cans
\item	MREs
\item	Clean and serviceable BII IAW the technical manual (-10)
\item	Complete first aid kits
\item	Any individual team equipment (i.e. can, mine detector, etc.)
\end{enumerate}
\item Vehicle Inspection:
\begin{enumerate}
\item	Vehicle is topped off and the fuel cans are properly secured to it.
\item	Air filters are clean.
\item	Suspension components are serviceable.
\item	Warning triangles checked for serviceability.
\item	Trailer properly hitched with LVC, safety chains. Canvas top checked for serviceability.
\item	Tires are inflated to the proper pressure on trucks and trailers.
\item	Basic Issue Item (BII)
\item	A completed PMCS and an available DA FORM 5988E.
\end{enumerate}
\item Communications:
\begin{enumerate}
\item	Radios are operational and set on the proper frequency.
\item	Ensure Timing is always set to DAGR Time in addition to checking frequencies.
\item	Secure equipment is functional and keyed.
\item	Matching units are set.
\item	Radio antennas are tied down.
\item	Connectors are clean and serviceable.
\item	Batteries are available for radios and fill devices
\item	Current TMs (-10s) are available.
\item	DA Form 5988E is on hand.
\item	OE 254s are complete.
\item	Spare batteries for equipment.
\item	Commo Plan card and SOI containing frequencies, call signs, challenge and passwords
\item	FM Radios are secure with locking bars and series 200 lock
NOTE: In addition to hardware accountability and serviceability inspections, communications require that frequencies, call signs, challenge, and passwords are disseminated.
\end{enumerate}

\item Miscellaneous Equipment:
\begin{enumerate}
\item	Tentage is clean, complete, and serviceable with camouflage
\item	Status boards are available
\item	Map boards are available
\item	Tables and chairs are loaded
\item	Brigade TACSOP on hand
\item	A reference library (such as ADP, ADRP, AR, FM, and TM) is available.
\item	Administration supplies are available that include:
\begin{enumerate}
    \item	Pens, pencils, etc.
\item Brigade OPORD/FRAGOs
\end{enumerate}
\end{enumerate}
\end{enumerate}
\end{enumerate}

\newpage
\section*{CARD 110 – TACTICAL UNIFORMS}
\begin{enumerate}
    \item General. This card prescribes uniform required in a tactical vehicle outside the cattle guard footprint. During field training, the Army Combat Shirt (ACS) may be worn in lieu of the ACU jacket as prescribed by the commander. The ACS will not be worn outside unit field training areas or ranges. Commanders may authorize Soldiers to wear the Patrol Cap while operating military wheeled vehicles in Garrison.

\item Level I Field Tactical Uniform
\begin{enumerate}
    \item ACUs/ OCPs/ UCPs/ boots
	\item ACH
	\item Eye protection
	\item Camelback
	\item Ear protection
	\item Gloves (carried, worn as needed)
	\end{enumerate}

\item Level II Field Tactical Uniform
\begin{enumerate}
    \item (Level I Tactical Uniform (+))
     \item	FLC complete including:
     \begin{enumerate}
         \item  1 x First Aid kit
\item 1 x Canteen (Configured per Soldier preference)
\item 3 x Ammunition Pouches
\end{enumerate}
\item	Weapon carried (M9 Holsters on chest, hip or thigh)
\item	Protective Mask (As Directed by Commander)
\item	MOPP 0-4 (As Directed by Commander) 
\end{enumerate}
\item Level III Deployment /Field Tactical Uniform
\begin{enumerate}
    \item 	ACUs/ OCPs/ UCPs/ boots
    \item	ACH
	\item Air Assault Pack with Camelback inside
    \item IOTV/ IBA/ TPC complete including:
    \begin{enumerate}
        \item First aid kit
        \item Front, Back and Side Plates
		\item 3 x Ammunition Pouches
		\end{enumerate}
	\item	Protective Mask
	\item	Weapon carried, M9 Holsters will be chest, hip or thigh
	\item	Eye protection, Ear protection, Gloves (carried, worn as needed)	\item	MOPP 0-4 (As Ordered) 
\end{enumerate}
\item Level IV Deployment/ Field Tactical Uniform
\begin{enumerate}
    \item Level III Tactical Uniform (+)
    \begin{enumerate}
        \item Neck Protector
		\item Groin Protector
		\item Shoulder Protector
		\item Side Plates
		\item Knee and Elbow Pads
		\end{enumerate}
		\end{enumerate}
\end{enumerate}

\newpage
\section*{CARD 111 – CBRN CONTAMINATION MARKING}
\begin{enumerate}
    \item General.   This card describes the contamination marking system used by the Company.

\item Markers.  The BN standard is NATO CBRN markers IAW FM 3-3 and FM 3-3-1.

\item Techniques.  
\begin{enumerate}
    \item Area Marking: When marking areas, ensure complete 360-degree coverage whenever possible, placing the markers at least 100 meters from the actual \item Equipment Marking: For supplies, vehicles, and equipment, whenever possible consolidate all contaminated materials at one location within the unit area and mark that location. However, decreasing the spread of contamination outweighs the need to consolidate material. If material is moved, units must check the movement route and mark that route as required.
\end{enumerate}
\item Not Marking Areas.   Subordinate commanders may elect not to post markers if a significant military advantage is obtained or if the area is to be relinquished to the enemy.  The BDE Commander must approve this decision and all friendly units warned of the contamination.
\item Reporting.  Use CBRN-4 Report
\begin{enumerate}
    \item Example - UPDATE
    \begin{enumerate}
        \item 
1)	Soldier Assigned with M16/M4.
a)	CTG 5.56mm ball			210
b)	CTG 5.56mm tracer			40
c)	Water 					5 qt

\item
2)	Soldier with M249.
a)	CTGg 5.56mm SAW			600
b)	Water					5 qt
\item
3)	Soldier with M9.
a)	CTG 9mm ball			30
b)	Water 					5 qt
\item
4)	Additional Equipment.   Platoon Leader may carry the following in addition to the above items.
a)	Green hd incendiary m14	  	2
b)	Green hd smoke hc			2
c)	Green hd smoke green     		2
d)	Green hd smoke red			2
e)	Green hd smoke yellow		2
f)	Green hd smoke violet		2
g)	Green star cluster			2
h)	White star cluster			2
i)	Optional equipment under CMDs discretion. 
\end{enumerate}
\end{enumerate}
\end{enumerate}

\newpage
\section*{CARD 112 – CONVOY COMMANDER CHECKLIST}
INSERT TABLE

\newpage
\section*{CARD 113 – WARNING ORDER	}
INSERT TABLE

\newpage
\section*{CARD 114 – ALERT PROCEDURES}
\begin{enumerate}
    \item Responsibilities
    \begin{enumerate}
        \item Battalion Commander: Initiates the recall, notifying and providing guidance to the Brigade.
\item	Battalion SDNCO/SDO: Contacts the staff primaries (OIC/NCOIC) and company command teams.
\item	Staff primaries: contact respective sections.
\item	Company command teams: contact respective subordinate leaders ensuring they know the exact time of N-Hour; report 100% notification to SDNCO
\item	Platoon Leaders: contact respective subordinate leaders.
\end{enumerate}
\item	Types of Alerts: The following code words, used to describe the actions Soldiers will take, will be used throughout the battalion for recall procedures.
\begin{enumerate}
    \item	Red Thunderbird - All personnel specified will report to their unit in duty uniform with field equipment, A bag, B bag and Ruck Sack no later than N+6.
\item	Copper Thunderbird - All personnel specified in the message will report to their unit in duty uniform no later than N+6.
\item	Brass Thunderbird - This is a telephonic recall notification test. Personnel will contact the next person on their alert roster. The last person contacts the first person on the list. The first person passes the results to the unit SDO/SDONCO. The unit SDO/SDONCO compiles the results reports them to the unit S3 who will report them to the Brigade EOC/DIVARTY/DIVISION CIOC.
\end{enumerate}
\item	Definitions
\begin{enumerate}
    \item Alert: Soldier or unit notification to begin assembly procedures in preparation for follow-on operations.
\item	N-hour: The time that initiates the N-hour sequence. This time is determined by higher and is specified during the alert process.
\item	N-hour sequence: A predetermined sequence of events that facilitate rapid unit deployment to a contingency operation. Typical tasks include recall of unit personnel, crisis action planning, final personnel readiness procedures, draw and preparation of equipment for deployment, and movement to the point of embarkation for immediate deployment.
\end{enumerate}
\item	N – Hour Sequence. 
The matrix below illustrates the battalion’s standard N-Hour sequence ISO 24 Hour PTDO from the time of initial notification: 
\item	Battle Tracking. 
Initial battle tracking during N-Hour sequence is handled by the BN SDNCO and SDO. It shifts to the S3 TAC upon establishment at N+5.5. At this point mission command is exercised through the BN TAC as well. See PACE plan above for amplifying detail.
\end{enumerate}

\newpage
\section*{CARD 115 – APEX / VISA / PASSPORTS}
INSERT IMAGE

\newpage
\section*{CARD 201 – COMSEC DESTRUCTION PROCEDURES}
\begin{enumerate}
    \item Destruction of Electronic Key. All electronic keys will be destroyed immediately upon supersession, if possible, but no later than 12 hours after supersession date.  Local commanders or responsible officials may grant extensions to a maximum of 72 hours on a case by case basis. Electronic keys that are destroyed will be annotated on the EKMW with the date of destruction, printed name, signature of the person who destroyed it, and a witness. 
\item Emergency Destruction Methods.
\begin{enumerate}
    \item 	It is the responsibility of the Commander to select appropriate destruction methods for a COMSEC Account. The Commander should base this selection on a comprehensive risk assessment and threat evaluation, conditions at each facility and available destruction methods.
\item	Priority of Destruction. Material will be destroyed utilizing the following priorities: 
\begin{enumerate}
    \item	Priority 1 – All superseded and current classified key marked “CRYPTO” in that order except authenticators, tactical operations codes, unused One Time Pads (OTP) and One Time Tapes (OTT). 
NOTE: Assume all electronic and physical key is CRYPTO unless marked otherwise. 
\item Priority 2 – TOP SECRET key that will become effective within the next 30 days
\item	Priority 3 – SECRET and CONFIDENTIAL key that will become effective within the next 30 days
\item	Priority 4 – Sensitive pages of crypto-equipment maintenance manuals (or the complete manual)
\item	Priority 5 – Classified components or sub-assemblies of COMSEC equipment (printed circuit boards and module boards) in the order listed in the appropriate maintenance manuals and then the remaining unkeyed CCI in accordance with the Commander’s CCI SOP.
\item	Priority 6 – The balance of the COMSEC equipment maintenance manuals and classified operating instructions
\item	Priority 7 – All remaining classified COMSEC material and unclassified key marked. “CRYPTO.” Destroy superseded authenticators and any remaining key, if time permits.
\end{enumerate}
\end{enumerate}
\\
INSERT TABLE
\end{enumerate}

\newpage
\section*{CARD 202 – COMSEC COMPROMISE PROCEDURES}
\begin{enumerate}
    \item Command Policy. Sensitive information and systems will be defended and secured from compromise, capture, and inadvertent disclosure to the greatest extent possible. Compromise of these items can jeopardize or disrupt operations, cause or increase casualties, and potentially endanger National Security. Every effort will be made to destroy these materials prior to capture or compromise. 
\item Compromise.  IAW TB 380-41, paragraph 6.1.1.a, compromise results from any event or action where COMSEC material is irretrievably lost or available information clearly proves that unauthorized persons have gained access to classified COMSEC information or unclassified key marked “CRYPTO”.
\begin{enumerate}
    \item COMSEC Incident.  IAW TB 380-41, paragraph 6.1.1.b, COMSEC incident is any occurrence that may potentially jeopardize the security of COMSEC material.  There are four types of COMSEC incidents. 
    \begin{enumerate}
        \item Physical Incident.  The loss, theft, capture, recovery by salvage, tampering, unauthorized viewing, unauthorized access, or unauthorized photography of classified COMSEC material or unclassified key marked CRYPTO.  The loss, theft, capture, recovery by salvage, or tampering with an unclassified keyed Controlled Cryptographic Item (CCI) is also considered a physical incident. 
\item Personnel Incident.  The fact that a person with access to or knowledge of cryptographic procedures and equipment is suspected of espionage, subversion or defection. 
\item Cryptographic incident.  Any equipment malfunction or crypto-operator error which may aid unauthorized persons to recover the plain text messages or the key to the cryptosystems. 
\item	Administrative incident.  Any significant occurrence that may jeopardize the integrity of COMSEC material or the information it protects as determined by the cognizant security authority.  Reportable only within the army chain-of-command as directed by the MACOM.  must be reported to the CONAUTH. 
\item	Actions which jeopardize the integrity of COMSEC material: 
\begin{enumerate}
    \item 	Premature or out-of-sequence keying.
\item	Inadvertent destruction of key material, as long as the destruction was properly performed and documented. 
\item	Destruction without authorization of the CONAUTH. 
\item	Failure to zeroize a common fill device or failure to destroy COMSEC material within required time limits.  Late destruction, over 30 days, is a reportable physical COMSEC incident. 
\item	Failure of the LE to properly safeguard or secure unclassified Cryptographic Ignition keys (CIK), PCMCIA cards, PKI Devices, Common Access Cards, etc. 
\item Loss of a KSV-21 user card.  
\item Destruction report errors (such as erroneous entries or missing information on a destruction report), including different dates recorded by the destruction official and witness (provided a pen and ink correction, as stipulated in paragraph “a” of this table, has occurred), a date block left blank, etc.
\end{enumerate}
\end{enumerate}

\item	Any other significant occurrence that may jeopardize the integrity of COMSEC material or the information it protects as determined by the cognizant security authority. 
\item	A COMSEC incident is not always a compromise.  The incident and facts related to its occurrence must be reported to the Controlling Authority (CONAUTH) for the material involved.  The CONAUTH will evaluate the incident and determine whether or not the incident is a compromise. 
\item	Specific questions about incident reporting and detection should be addressed to 86th Signal Battalion CMO. 
\end{enumerate}
\item	Reporting Incidents (per regulation). 
\begin{enumerate}
    \item 	Director USACSLA, Director NSA, or other responsible authorities must decide whether or not an actual compromise has occurred. 
\item	The decision and actions taken to lessen the impact of compromises are based on a report of incident.  The detection and reporting of incidents are the responsibilities of every individual who possesses, handles, or uses COMSEC information to include all supervisors whether involved directly or indirectly. \item Commanders, supervisor, or users are not authorized to make a determination as to whether or not a compromise has occurred but are responsible for reporting the act which resulted in an incident.
\item	Individual users or the person detecting a COMSEC incident must report incidents or possible compromises to their next higher superior in the chain of command, their Commander, and the 86th Signal Battalion KOAM.  Reporting to only the security officer/manager does not meet this requirement.  COMSEC incidents are reported and handled by a different system than that used for NON-COMSEC classified materials and information. 
\item	Reports of incidents must contain all available information to include the following: 
\begin{enumerate}
\item	Complete identification of the material involved. 
\item	Description of the incident. 
\item	Date and circumstances of last sighting, or exact period of time and under what circumstances the material was discovered to be out of the proper channels. 
\item	Identity and clearance status of ALL personnel involved. 
\item	Length of time unauthorized personnel had access to the material or length of time that the material was left unprotected.
\end{enumerate}
\end{enumerate}
\end{enumerate}

\newpage
\section*{CARD 203 – MEDEVAC PROCEDURES}
\begin{enumerate}
    \item 	Medical Planning Considerations
    \begin{enumerate}
        \item Casualty Collection Points (CCPs) are planned at company and platoon level.  Platoon CCPs are controlled by the platoon sergeant.  The PSG is the primary means of evacuation for all platoon casualties.  The platoon leader’s vehicle is the alternate evacuation vehicle.  Vehicles will move to platoon CCPs then to company/troop CCPs.
\item	Crew members/combat lifesavers/medics will administer first aid to injured soldiers and report through platoon leader/sergeant to company CP/1SG to request evacuation.
\item	The injured soldier’s Kevlar, LCE, sleeping bag, personnel hygiene bag, MRE, dog tags, MARC card, ID Card, DA 1155, DA 1156, and protective equipment will accompany him.  Turn in all remaining equipment to the PSG for turn-in.
\item	Casualties are transported from the scene of action to the troop/company combat trains/casualty collection point by the company ambulance or platoon 	vehicles, when possible.  Urgent casualties are transported directly to an aid station or ambulance exchange point.  Contaminated casualties are evacuated on “dirty” routes indicated on the CSS overlay.
\item	Priority of evacuation.
		(1)	Green – Routine		6 Hours
		(2)	Orange – Priority		2 Hours
		(3)	Red – Urgent			1 Hour
		(4)	Yellow – CBRN		30 Min
		(5)	KIAs as soon as possible (mission dictates), or as requested via FM.
\item	Air MEDEVAC is utilized only if the casualty is at risk of losing life or limb, or if the tactical situation demands its’ use.
\item	Location and movement of battalion aid stations is critical to rapid treatment, stabilization, and survival of casualties.  Aid stations must be located as far forward as possible for rapid transport and treatment.  Soldiers must know the bumper number/unit reference number (URN) of aid stations so that they can rapidly find them on the JBC-P situational display.  Whenever 	possible, the FSB medical company will co-locate ambulance exchange points (AXPs) with aid stations to reduce turnaround times for company ambulances and to reduce patient handling/transfer.
\item	The primary FM frequency for controlling medical evacuation across the Brigade is the forward support medical company command frequency.  This frequency will be entered into the JBC-P MEDEVAC message as part of platoon PCCs/PCIs.  
 \end{enumerate}
\item	Calling a 9-line
\begin{enumerate}
    \item Do not assume you will never come into contact simply because you are not in a combat MOS. Signal Soldiers must never neglect critical tactical skills, including being able to provide medical aid for another. An important aspect of providing medical assistance is your ability to accurately send a MEDEVAC/CASEVAC request. Understanding basic radio procedures is necessary, as sending an accurate MEDEVAC request could mean saving someone’s life. 
\item In addition to understanding radio procedures and how to send this report, deployed leaders should ensure their Soldiers know about proper aid procedures in the event of a medical emergency on their base. Knowing where mass casualty (MASCAL) bags and casualty collection points (CCPs) are on your FOB/COB is vital. Understanding what actions to take in a medical emergency must be a priority for all deployed Soldiers.
\end{enumerate}
\\
INSERT IMAGES with numbering titled
\\
\item	Civil Disturbance Checklist
\begin{enumerate}
    \item Purpose. This prescribes the considerations platoon soldiers will follow in reacting to a civil disturbance.
\item	General. Belligerent parties and gangs may start civil disturbances or riots.  In some cases, their intent may be to provoke a level of response by US forces that may be exploited politically and in the media.
\begin{enumerate}
    \item 	Commander’s intent and the ROE will always govern actions when any unit, regardless of how small, reacts to a civil disturbance.
\item	Host nation officials, police, MPs, US or allied CA, and PSYOP must be involved as early as possible in maintaining or restoring order in the population.
\item	Request to use non-lethal munitions (such as tear gas) begins with the TF Commander.
\end{enumerate}
\item	Actions: When civil disturbances occur the platoon will:
\begin{enumerate}
    \item	Identify grid coordinates of front line of disturbance and size; report.
\item	Maintain distance from disturbance but continue observation; reporting movements, growth and shrinkage of the disturbance size.
\item	Maintain 360o security, wait to be relieved or ordered to be moved on.
\end{enumerate}
\end{enumerate}
\end{enumerate}

\newpage
\section*{CARD 321 – INTELLIGENCE BRIEF}
\begin{enumerate}
    \item  General. To outline the types of briefs required by the S2 to give to deploying teams prior to departure for mission.
\item Responsibilities
\begin{enumerate}
    \item All personnel are required to receive a culture awareness and country threat briefs prior to deployment to ensure readiness during travel to a mission location.
\item Cultural competence is defined as the ability to interact effectively with people of different cultures and socio-economic backgrounds, particularly in the context of human resources, non-profit organizations, and government agencies whose employees work with persons from different cultural/ethnic backgrounds.
\end{enumerate}
\item  Culture Awareness Brief
\begin{enumerate}
    \item Cultural awareness leads to an understanding of how a person’s culture can inform their values, behavior, beliefs, and basic assumptions.
\item Cultural awareness recognizes that we are all shaped by our cultural background, which influences how we interpret the world around us, as well as, perceive ourselves and relate to other people.
\item Culture awareness will cover the following:
\begin{enumerate}
    \item Religion
\item	Cultural norms/ Beliefs
\item	Race and Ethnicities 
\item	OMB Race and Ethnic Groups
\end{enumerate}
\end{enumerate}
\item Country Threat Brief 
\begin{enumerate}
    \item  All deploying personnel will be required to attend a Country threat brief. A Country Threat Brief will give an overview of the initial threats in the area and how they will impact the overall ability to accomplish the mission. 
    \item A Country threat brief will cover the following:
    \begin{enumerate}
        \item 	Enemy capabilities
\item Electronic Warfare
\item	Cyber Threats
\item Terrain and Weather
 \end{enumerate}
EXAMPLE INTEL BRIEF HYPERLINK
\end{enumerate}
\end{enumerate}

\newpage
\section*{CARD 501 - 11TH TTSB HQ CALL SIGNS }
INSERT IMAGE/TABLE

\newpage
\section*{CARD 502 – 11th TTSB 40TH ESB CALL SIGNS}
INSERT IMAGE/TABLE

\newpage
\section*{CARD 503 – 57TH ESB CALL SIGNS}
INSERT IMAGE/TABLE

\newpage
\section*{CARD 504 – 62ND ESB CALL SIGNS}
INSERT IMAGE/TABLE

\newpage
\section*{CARD 505 – 86TH ESB CALL SIGNS}
INSERT IMAGE/TABLE

\newpage
\section*{CARD 701 – WEAPONS AND AMMUNITION SAFETY}
\begin{enumerate}
    \item 	Weapon Handling.
    \begin{enumerate}
        \item 	Horseplay will not be tolerated.
b.	Ensure weapons are kept on safe.
c.	Remind Soldiers to consider weapons “LOADED” at all times and to check chamber often.
d.   Never point a weapon at anything you do not intend to shoot.
e.   Keep you trigger finger straight and off the trigger until ready to fire.
f.	Train in target identification.
g.	Control Ammo.
h.	Highlight danger of "cook off".
i.	Rehearse immediate-action drills for misfires/weapons malfunctions.
j.	Require that weapons, ammo and magazines be kept clean.  Muzzles will be covered to prevent clogging.
\end{enumerate}
\item	Ammunition and Explosives.
\begin{enumerate}
    \item Expose only the minimum number of people and amount of equipment necessary.
b.	Handle ammunition carefully.
c.	Coordinate with QM laundry to wash clothing with an anti-static additive to reduce static electricity.
d.	Don't use sparking metallic tools on explosives.
e.	Determine if your area of operations is susceptible to electrical storms and establish lighting protection procedures.
f. 	Keep all flammable materials and all flame or spark producing devices away from ammunition and explosives.
g.	Ensure fire extinguishers are present.
h.	In case of fire, evacuate the area to a distance of at least 400 meters and take cover.
i.	Ensure vehicle brakes are set, engine is turned off, and at least one wheel is chocked during loading and unloading.
j.	Ensure vehicles and trailers loaded with ammunition are parked at least 50 feet from vehicles and trailer loaded with flammable liquids.
k.	Protect ammunition from direct sun.  Allow for ventilation.
l.	Disperse ammunition to minimize loss in the event of fire, accidental explosion or enemy action.
m.	When storing ammunition, use sand dunes, barriers, and building materials to protect personnel and material from the effects of explosion.
\end{enumerate}
\end{enumerate}

\newpage
\section*{CARD 702 – COMMUNICATIONS AND ANTENNA SAFETY}
\begin{enumerate}
    \item   Communications Safety.
    \begin{enumerate}
        \item Follow all cautions in the TMs.
	b.	Antennas. Make sure to employ the appropriate number of personnel when erecting all antennas. 
	\begin{enumerate}
	    \item (1)	Remind personnel that when erecting antennas, they must stay twice the distance from power lines as the length of the antenna.
(2)	Stress that falling antenna head sections have killed Soldiers.
(3)	Require that personnel wear eye protection, head protection and gloves when erecting antennas.
(4)	Do not allow substitutes for antenna mast sections.
(5)	Guard or cordon off any antenna head that is on the ground to prevent anyone from walking in it.
(6)	Antenna tips must be installed.
	\end{enumerate}

	\item c.	Power Lines.
	\begin{enumerate}
\item (1)	Identify power lines in operational areas to all Soldiers.
(2)	Tie down antennas when in areas of power lines. Antenna tip should be no lower than 7 feet to preclude eye injuries.
(3)	Warn Soldiers to never throw WD1/WF-16 over power lines.
\end{enumerate}
	\item d.	Electrical Storms.
(1)	If possible, do not operate radios, telephone, or switchboards.
(2)	Disconnect electrical equipment from power sources and antennas if the situation 				permits.
(3)	If the equipment must be used, converse as little as possible.  Return call after the 				storm.
\item	e.	Grounding.
(1)	Remind personnel that extra care must be given to preventing static electricity in hot, dry climates.
(2)	Ensure personnel know any special grounding procedures in accordance with FM 20-31,
	 Engine-Driven Electrical Generator Sets.
(3)	Remind personnel to ground themselves by touching a large metal object before handling fuel hoses and nozzles.
(4)	Ensure grounding and bonding equipment is inspected regularly.
\end{enumerate}
\item  Antenna Safety. During installation and use of vehicular whip antennas, field type masts, towers, antennas, and metal poles, damage to equipment, serious injury, and death could result when personnel fail to observe safety precautions. 
	a.	The technical manuals (TM) and field manuals (FM), provided for each antenna, should be thoroughly read before attempting any installation.
	b.	Precautions to be followed in order to prevent serious injury or death are outlined below:
	(1) Whip antennas:
(a)	Never lean against or grasp a whip antenna when the transmitter is operating, severe burns may result.
(b)	When operating with vehicular equipment, never pass under power lines unless there is adequate overhead clearance between the lines and the antenna.
(c)	Unless mobile operation is mandatory, always tie down the antenna so the distance from the tip cap to the ground is seven feet or less.
(d)	Whip antennas will have a protection tip cap to prevent personal injury. Caps can be secured with a low temperature tape such as black electrical tape.
(2)	Masts, Towers, and Antenna Assemblies:
(a)	Masts, towers and antenna assemblies must be installed as far away from power lines as possible. As a basic guide, a distance equal to at least twice the height of the structure should be maintained.
(b)	Guy wires will be kept as far away from power lines as possible, thereby eliminating the possibility of a power line falling across the guy.
(c)	Tall structures should have adequate lightning protection as prescribed in the TM or as necessary according to regulatory requirements.
(d)	Never engage in work on a mast, tower, or antenna during an electrical storm or when one is imminent.
(e)	All personnel engaged in the installation of antennas are required to wear Kevlar helmets, eye protection, and gloves.
(f)	Mark off the area around the base of the antenna at least waist high with engineer tape for daylight operations and chem. lights for night operations. Markings should also restrict vehicle access.
(g)	Never climb an installed/erected mast or tower to make any adjustments.
\end{enumerate}

\newpage
\section*{CARD 706 – ACCIDENT REPORTING}

\section{Conclusion}
``I always thought something was fundamentally wrong with the universe'' \citep{adams1995hitchhiker}

\bibliographystyle{plain}
\bibliography{references}
\end{document}

